\documentclass[a4paper,12pt]{ctexart}%有了这个ctexart的模板,使用中文是很方便的。
\begin{document}
\pagestyle{empty}
\begin{center}
\huge \textbf{VIM}
\end{center}

\newpage
\tableofcontents

\newpage
\section{vim介绍}
Vim是从 vi 发展出来的一个文本编辑器。代码补全、编译及错误跳转等方便编程的功能特别丰富,在程序员中被广泛使用,和Emacs并列成为类Unix系统用户最喜欢的文本编辑器。

\section{vim配置介绍}
vim的大部分设定都位于\verb|~/.vimrc|文件中。如果没有该文件可以创建一个文件。如果要添加鼠标支持,添加缩进设置,可以添加对应的命令。

\section{鼠标支持}
vim是可以添加鼠标支持的,即使是SSH连接到远程服务器也是可以的,但是默认是没有激活的,其实我用过之后,我认为也不需要激活。对新手而言,激活的方法有两种:
\begin{enumerate}
  \item 进入vim的时候,执行\verb|:set mouse=a|
  \item 将\verb|set mouse=a|命令添加到\verb|~/.vimrc|中。
\end{enumerate}

\section{模式介绍}
就我的水平而言,目前使用两种模式。
\subsection{命令模式}
就是进入vim时候的模式,左下角没有任何显示,最基本的模式。
\subsection{编辑模式}
在命令模式的时候,左下角显示--INSERT--按下相关的命令(如:i)进入编辑模式。这个时候就像在其他的文本编辑器下面一样进行操作即可。进入编辑模式的命令如下:
\begin{description}
  \item[a] 这个是我常用的按键,在当前字符的后插入。
  \item[i] 这个也是我常用的按键,在当前位置插入。
\end{description}
这个命令暂时就介绍两个,其他的就不介绍了。

\section{移动光标}
关于光标的移动有多种方法,分为两类,鼠标操作,键盘操作。
\subsection{鼠标操作}
vim的鼠标功能默认是没有激活的,要手动激活。

在命令模式下输入\verb|:set mouse=a|,这是激活的方法。当然将这条命令加入\verb|.vimrc|文件可以方便很多,只要设置一次就行。
\subsection{键盘操作}
键盘操作相对比较复杂。在编辑模式下移动光标很简单,使用上下左右键,就不讲了。下面讲的是在命令模式下移动光标的方法
\begin{description}
  \item[Ctrl+f] 向下移动一屏
  \item[Ctrl+d] 向下移动半屏
  \item[Ctrl+b] 向上移动一屏
  \item[Ctrl+b] 向上移动半屏
  \item[gg]     跳到第一行
  \item[G]      跳到最后一行
\end{description}

\section{查找和跳转}
查找的命令是\verb|:/查找内容|,按n键查找下一个,按N键查找上一个。如果希望搜索到的结果高亮,需要在命令模式下输入 \verb|:set hlsearch|,当然了,加入到\verb|.vimrc|文件中是最好的选择。
\begin{verbatim}
如果一个单词比较长,可以将光标移动到该单词上,按下*或者#进行该单词的搜索。
(*相当于/搜索,#相当于?搜索)
\end{verbatim}
跳转到121行的命令是\verb|:121|,很简单。

\section{删除和撤销}
删除既可以在命令模式下完成,也可以在编辑模式下完成。在编辑模式下很简单,在这里我们主要讲在命令下删除或者撤销的方法。
\subsection{删除相关}
删除主要的就是删除当前字符和删除当前行。还有一些其他的技巧性的操作。\verb|$|的意思一般是指到最后的意思。在面可以感受到。
\begin{verbatim}
  x     删除当前字符,即光标所在的地方的字符。
  3x    删除当前的三个字符,即光标向后三个字符。
  d$    删除当前字符之后的所有字符。
  D     删除当前字符至行尾。
  dd    删除当前行。
  10d   删除当前行开始的10行。
  :4,10d    删除4到10行。
  :4,$d     删除4行之后的所有行。
\end{verbatim}

\subsection{撤销}
撤销和重做(撤销的撤销)
\begin{description}
  \item[u] 向前撤销一个操作
  \item[ctrl+r] 撤销之前那个撤销的动作。
\end{description}

\section{自动完成}
和notepad++一样,vim当然有自动补全的功能。输入字符串的前面部分字符,然后按ctrl+p

\section{替换命令}
替换操作是一个常用的操作。这里只介绍三个,其他的可以不用知道。
\begin{verbatim}
  r     替换当前字符,如ra命令将当前的字符替换成r。
  %s/old/new/g  在全文范围内将old替换成new。
  ddp   交换光标所在行和其下紧邻的一行。
\end{verbatim}









\section{退出命令}
\begin{verbatim}
  ZZ    保存并退出。
  :wq   保存并退出。
  :q!   不保存退出。
\end{verbatim}
\end{document} 