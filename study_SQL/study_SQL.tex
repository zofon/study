\documentclass[a4paper,12pt]{ctexart}
\usepackage[margin=2cm]{geometry}
\usepackage{graphicx}
\usepackage{subfigure}
\usepackage{float}
\usepackage[colorlinks,linkcolor=black]{hyperref}%colorlinks启用链接颜色,linkcolor指定对应的颜色
\pagestyle{empty}
\CTEXsetup[format={\Large\bfseries}]{section}%可以让section的标题左对齐。
%\CTEXsetup[format+={\flushleft}]{section}%让section的标题居左
%\renewcommand{\thesection}{\chinese{section}}%将“1.1”改为汉字“一”,但是subsection就会变成  六.1 ,比较难看,还是不用比较好。
\begin{document}

\begin{center}
\huge \textbf{数据库}
\end{center}

\section{数据库理论}
\section{引言}
当前的数据库类型有上千种,目前和Oracle并列的还有Microsoft和Adobe。

数据库最早的来源于一片论文

\section{SQL}
SQL(Structured Query Language)结构化查询语言。
\subsection{模式的定义与删除}
定义SQL模式,以S\_SC\_C为例:
CREATE SCHEMA <模式名> AUTHRIZATION <用户名> [<CREATE DOMAIN 子句>|<CREATE TABLE 子句>|<CREATE VIEW 子句>|...]\\
其中[ ]中的内容是可选项。

例:CREATE SCHEMA S\_SC\_C AUTHRIZATION Jin;

删除SQL模式:
DROP SCHEMA <模式名> {CASCADE|RESTRICT}\\
{}里面的内容表示二选一

例: DROP SCHEMA S\_SC\_C CASCADE

\subsection{基本表的定义、删除与修改}
基本表的定义:

\begin{verbatim}
CREATE TABLE S\_SC\_C.STUDENT           #在S_SC_C这个SQL模式下添加一个表。
(
    s#      CHAR(8),
    sname   CHAR(20)    NOT NULL UNIQUE,      #不能出现空值,且唯一
    sex     CHAR(2)     NOT NULL DEFAULT '男', #不能出现空值,且默认值为‘男’
    age     INT,                              #可以为空,且默认值为NULL
    dept    CHAR(20),                         ##可以为空,且默认值为NULL
    PRIMARY KEY(s#)
);
\end{verbatim}
同理,下面创建一个SC的基本表:
\begin{verbatim}
CREATE TABLE S\_SC\_C.SC
(
    s#      CHAR(8),
    c#      CHAR(8),
    grade   INT,
    PRIMARY KEY(s#),
    FOREIGN KEY(s#)REFERENCES STUDENT(s#),
    FOREIGN KEY(c#)REFERENCES COURSE(c#)
);
\end{verbatim}


删除基本表:
\verb|DROP TABLE S_SC_C CASCADE|


\section{Oracle实践}
\section{Oracle学习}
\subsection{Oracle安装}
准备出5G的空间。
Oracle 8i表示向网络方向发展
Oracle 9i是Oracle 8i的稳定版,3CD。
Oracle 10g表示向网格计算(更高效的搜索算法)发展。
Oracle 11g是10g的稳定版,是最主流的广泛。
Oracle 12C,最新的版本是,表示云计算(cloud),本次将课使用11g版本。

防止日后的程序乱码,字符集选择UTF-8
\begin{figure}[H]
  \centering
  \includegraphics[width=10cm]{oracle/安装步骤_指定配置_选择字符集.png}
%  \caption{}\label{}
\end{figure}
接下来事例方案选项卡中选择创建具有事例方案的数据库。比较重要。
\begin{figure}[H]
  \centering
  \includegraphics[width=10cm]{oracle/安装步骤_指定配置_选择创建示例方案.png}
%  \caption{}\label{}
\end{figure}

接下来是配置口令,不同的管理员可以设置不同的密码,但是现在我们设置一样的密码,我的密码是241833Ab。

四个主要的用户
\begin{itemize}
  \item sys 超级管理员
  \item system 普通管理员
  \item scott 普通用户
  \item sh 大数据用户
\end{itemize}

\begin{figure}[H]
  \centering
  \includegraphics[width=10cm]{oracle/安装步骤_指定配置_配置口令.png}
%  \caption{}\label{}
\end{figure}

然后就是环境检测,即使环境检测结果有问题也没有关系,可以跳过,进行安装。安装的时候可能会弹出一些错误点击忽略。记得拖滚动条选择SCOTT账号,设置密码。
安装完成后进行口令管理:
\begin{figure}[H]
  \centering
  \includegraphics[width=10cm]{oracle/口令管理.png}
%  \caption{}\label{}
\end{figure}
最后点击安装完成。

\section{SQLPLUS基本命令}
打开控制台,输入sqlplus.exe,然后输入账号scott和密码fe:
\begin{figure}[H]
  \centering
  \includegraphics[width=10cm]{oracle/打开sqlplus.png}
%  \caption{}\label{}
\end{figure}
格式调整命令:
\begin{verbatim}
SQL> select * from emp;     #不区分大小写,此时显示不清楚,有断行
SQL> set linesize 300;      #设置每行的宽度
SQL> set pagesize 30;       #设置每页的行数
\end{verbatim}

调用记事本命令(将命令打包):
\begin{verbatim}
SQL> ed 文件名称                                    #创建一个记事本,将命令放在里面,批量执行
SQL> @文件名.sql             #执行该文件
SQL> @d:\文件名.sql          #执行磁盘上的某个数据库脚本文件
\end{verbatim}

用户管理命令:
\begin{verbatim}
SQL> show user              #显示当前用户
SQL> conn sys/fe as sysdba  #切换到sys这个用户,fe是密码
SQL> conn scott/fe          #切换回scott
\end{verbatim}

调用本地机器命令,但是要在之前加上host.
\begin{verbatim}
SQL> host mkdir test
\end{verbatim}
\section{引用的例子}
这是一个引用~\cite{LCN2002}

\section{SQL语法}
简单查询
限定查询
查询排序

scott用户的主要数据表(***)。  一个数据库中有大量的表。
\begin{verbatim}
SQL> select * from tab;     #查询某个用户下的所有表
SQL> desc dept              #查看表的结构
SQL> select * from dept;    #查看表的内容
SQL> select * from emp;
SQL> select * from salgrade;
\end{verbatim}
\begin{figure}[H]
  \centering
  \includegraphics[width=6cm]{oracle/table_dept.png}
%  \caption{}\label{}
\end{figure}
\begin{figure}[H]
  \centering
  \includegraphics[width=14cm]{oracle/table_emp.png}
%  \caption{}\label{}
\end{figure}
\begin{figure}[H]
  \centering
  \includegraphics[width=5cm]{oracle/table_salgrade.png}
%  \caption{}\label{}
\end{figure}

DML(**),又称为DQL
DDL(**)
DCL,一般由DBA负责。

简单查询,限定查询,多表查询,统计查询四类查询

简单查询,显示全部的数据行。



\bibliographystyle{plain}%
\bibliography{Referenzarchiv}

%\bibliographystyle{IEEEtran.bst}
%\bibliographystyle{plain}
%表示指定文献引用的格式设置参考文献的类型 (bibliography style). 标准的为 plain:
%其它的类型包括:
%unsrt – 基本上跟 plain 类型一样, 除了参考文献的条目的编号是按照引用的顺序, 而不是按照作者的字母顺序.
%alpha – 类似于 plain 类型, 当参考文献的条目的编号基于作者名字和出版年份的顺序.
%abbrv – 缩写格式 .

%\bibliography{ReferenzarchivWithoutURLs,OtherReferences}
%对应的引用文件。
\end{document}
