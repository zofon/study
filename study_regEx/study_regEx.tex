\documentclass[a4paper,12pt]{ctexart}
\usepackage[margin=2cm]{geometry}
\usepackage{graphicx}
\usepackage{subfigure}
\usepackage{float}
\usepackage{url}
\usepackage[colorlinks,linkcolor=black]{hyperref}%colorlinks启用链接颜色,linkcolor指定对应的颜色
\pagestyle{empty}
\CTEXsetup[format={\Large\bfseries}]{section}%可以让section的标题左对齐。
%\CTEXsetup[format+={\flushleft}]{section}%让section的标题居左
%\renewcommand{\thesection}{\chinese{section}}%将“1.1”改为汉字“一”,但是subsection就会变成  六.1 ,比较难看,还是不用比较好。
\begin{document}

\begin{center}
\huge regEx
\end{center}
\normalsize

%\tableofcontents
%\newpage
\section{介绍正则表达式}
正则表达式用于匹配某些复杂规则的字符串,换句话说,正则表达式就是记录文本规则的代码。如:用于文件查找的通配符*和?,*会被解释成任意的字符串。这是最简单的正则表达式~\cite{regEx入门}。


\section{专业术语}
\begin{itemize}
  \item 通配符(wildcard)
  \item 元字符(metacharacter)
\end{itemize}

\section{例子}
\subsection{匹配电话号码}
正常的电话号码是0523-84637261,前面的区号可能3位或者4位,这是最基本的了。
解答:\verb|\d{3,4}-\d{8}|
\subsection{匹配IP地址}
这是一个经典的问题了,当第一位为2的时候,第二不能为6以上,第二位为5时,第三位不能为6以上。要解决这个问题,我们需要用到分枝条件。

解答:
\begin{verbatim}
((2[0-4]\d|25[0-5]|[01]?\d\d?)\.){3}(2[0-4]\d|25[0-5]|[01]?\d\d?)

理解这个表达式的关键是理解是:2[0-4]\d|25[0-5]|[01]?\d\d?
\end{verbatim}

\subsection{搜索文本中的内容并导出}
在文件file.txt中搜索'target.data'之后('target.data'可以是正则表达式),导入到out.log
\begin{verbatim}
grep 'target.data' file.txt | tee -a out.log
\end{verbatim}

\bibliographystyle{plain}
\bibliography{Referenzarchiv}

%\bibliographystyle{IEEEtran.bst}
%\bibliographystyle{plain}
%表示指定文献引用的格式设置参考文献的类型 (bibliography style). 标准的为 plain:
%其它的类型包括:
%unsrt – 基本上跟 plain 类型一样, 除了参考文献的条目的编号是按照引用的顺序, 而不是按照作者的字母顺序.
%alpha – 类似于 plain 类型, 当参考文献的条目的编号基于作者名字和出版年份的顺序.
%abbrv – 缩写格式 .

%\bibliography{ReferenzarchivWithoutURLs,OtherReferences}
%对应的引用文件。
\end{document}
