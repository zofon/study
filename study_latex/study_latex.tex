\documentclass[a4paper,12pt]{ctexart}
\usepackage[margin=2cm]{geometry}
\usepackage{graphicx}
\usepackage{subfigure}
\usepackage{float}
\usepackage[colorlinks,linkcolor=black]{hyperref}%colorlinks启用链接颜色,linkcolor指定对应的颜色
\usepackage{listings}
%\usepackage[cache=false]{minted}  % 代码高亮X
\pagestyle{empty}
\CTEXsetup[format={\Large\bfseries}]{section}%可以让section的标题左对齐。
%\CTEXsetup[format+={\flushleft}]{section}%让section的标题居左
%\renewcommand{\thesection}{\chinese{section}}%将“1.1”改为汉字“一”,但是subsection就会变成  六.1 ,比较难看,还是不用比较好。
\begin{document}

\begin{center}
\huge \textbf{latex学习}
\end{center}

\tableofcontents

\newpage
\section{最简单的模板}
请参考当前目录下的DEMO.tex文件。




\section{关于编辑器和编译平台}
在windows环境下,ctex是最好的选择,然后使用WinEdt作为编辑器最好。

在linux环境下,编辑器一般是vim。要安装tex相关的软件。命令如下:
\begin{verbatim}
sudo apt-get install texlive-full
\end{verbatim}




\section{定义新的函数}
如下的函数,在添加第一行之前是不能使用的,因为在并没有avg这个函数,所以需要添加对应的定义。
\begin{verbatim}
\def\avg{\mathop{\hbox{avg}}}
\begin{small}
\begin{equation}
B \ge
   \left( 3 * \avg_{1 \leq i \leq n} \{\mathrm{RTT}_{i}\} +
              \max_{1 \leq i \leq n} \{\mathrm{RTO}_{i}\} \right) *
   \sum_{i=1}^{n}{\mathrm{Bandwidth}_{i}}.
\label{eq:practicle-size-1}
\end{equation}
\end{small}
\end{verbatim}




\section{上标和下标}
如:$a^2 + b^2 = c^2$的表示方法是:
\begin{verbatim}
a^2 + b^2 = c^2
\end{verbatim}
如果上标比较复杂,则将上标部分使用大括号括起来。如$a^{x^2+y^2}=c$的代码是:
\begin{verbatim}
a^{x^2+y^2}=c
\end{verbatim}
下标和上标的原理差不多,如$a_{x^2+y^2} = d$代码是:
\begin{verbatim}
a_{x^2+y^2} = d
\end{verbatim}




\section{关于图片的位置}
添加包:usepackage{float},图片对应的位置是由后面的参数决定的。
例子如下:
\begin{verbatim}
\begin{figure}[H]
  \centering
  \includegraphics[width=15cm]{Figures/Numix_theme_1.jpg}
\end{figure}
\end{verbatim}




\section{关于原文抄录}
一般来说使用verbatim命令和verb命令:
\subsection{verbatim}
\begin{verbatim}
\begin{verbatim}
文字,如果添加*的意思就是空格以下划线的形式输出。
\end{verbatim}
\end{verbatim}
\subsection{verb}
verb不会换行,一般用于简短的抄录,用法如下:
\begin{verbatim}
\verb|文字|
\verb*|文字|  添加*的意思就是空格以下划线的形式输出。
\end{verbatim}



\section{嵌入代码块}
就目前为止,最好的嵌入代码块的方式是使用minted包。这个包要用到python的一个插件。
\subsection{安装相关的包}
在windows下的时候比较麻烦,这个东西需要安装python和一些其他的东西,自己上网搜。

在linux下的时候可以通过如下命令安装:\verb|sudo apt install python3-pygments|

\subsection{使用方法}
在导言区添加:\verb|\usepackage{minted}|

在文中使用的方法如下:
\begin{verbatim}
\begin{minted}{c++}
int main() {
    printf("hello, world");
    return 0;
}
\end{minted}
\end{verbatim}


\section{标题居左}
往导言区添加如下命令:
\begin{verbatim}
\CTEXsetup[format={\Large\bfseries}]{section}%可以让section的标题左对齐。
\end{verbatim}


\section{关于文献的引用}
文献的引用有几种方法,在windows下的时候和linux不一样:
\subsection{使用Bitex方法}
首先编写bib文件,这个文件编写之后在用之前是要编译一下的。

命令如下:\verb|bibtex FengZhou-MPTCP|

然后在文档的下方加入如下命令:
\begin{verbatim}
\bibliographystyle{IEEEtran.bst}%表示指定文献引用的格式
\bibliography{ReferenzarchivWithoutURLs,OtherReferences}%对应的引用文件。
\end{verbatim}

\subsection{引用的方法}
然后在要引用的地方输入\verb|\cite{Referenzarchiv}|,即可,如:~\cite{LCN2002}。

latex编译一次, bibtex 编译一次, 再用 latex编译两次就大功告成了

注:在文章中至少要引用一篇文献才行。





\bibliographystyle{plain}
\bibliography{Referenzarchiv}

%\bibliographystyle{IEEEtran.bst}
%\bibliographystyle{plain}
%表示指定文献引用的格式设置参考文献的类型 (bibliography style). 标准的为 plain:
%其它的类型包括:
%unsrt – 基本上跟 plain 类型一样, 除了参考文献的条目的编号是按照引用的顺序, 而不是按照作者的字母顺序.
%alpha – 类似于 plain 类型, 当参考文献的条目的编号基于作者名字和出版年份的顺序.
%abbrv – 缩写格式 .

%\bibliography{ReferenzarchivWithoutURLs,OtherReferences}
%对应的引用文件。
\end{document}
