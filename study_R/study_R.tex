\documentclass[a4paper,12pt]{ctexart}
\usepackage[margin=2cm]{geometry}
\usepackage{graphicx}
\usepackage{subfigure}
\usepackage{float}
\usepackage[colorlinks,linkcolor=black]{hyperref}%colorlinks启用链接颜色,linkcolor指定对应的颜色
\pagestyle{empty}
\begin{document}

\begin{center}
\huge \textbf{R学习}
\end{center}

\tableofcontents
\newpage

\section{R语言介绍}
R语言和其他的语言之间提供了非常好的接口。
\begin{itemize}
  \item R语言对大小写敏感。
  \item 基本的命令是表达式或者赋值。
  \item 命令可以被;隔开。
  \item 注释符号用\#
  \item 
  \item 
  \item 
\end{itemize}

%暂时没有理解这句话。
R的缺点:
\begin{itemize}
  \item 耗内存,所以要用rm命令来删除对象,以释放内存,如:\verb|rm(x,y,z)|。
  \item 精度有问题
\end{itemize}

\section{R命令介绍}
在linux下面使用R的时候,我们一般用到的命令是:
\begin{verbatim}
R --vanilla <plot.R > a.out
\end{verbatim}
这里的 --vanilla 是参数,当然还有别的很多参数,具体有兴趣可以用man命令查看。
<plot.R 的意思是将plot-R这个文件作为一个输入。既然有输入,那自然有输出,>a.out 的意思就是将显示的内容输出到a.out这个文件中。

\section{关于包的安装与使用}
\subsection{安装对应的包}
R语言中,有多种方式可以导入数据包,如下,一目了然:
\begin{verbatim}
library(Hmisc)
source("plotter.R")
\end{verbatim}
如果没有对应的包,就要安装,其中source("plotter.R"),这个plotter.R文件是在当前的目录下的。
在linux下,进入R之后,用如下命令:
\begin{verbatim}
install.packages("ggplot2")
install.packages("Hmisc")
\end{verbatim}
\subsection{包的使用}
用source命令导入包,如:\verb|>source("plot.R")|

\section{查看帮助}
如查看solve的帮助:
\begin{itemize}
  \item \verb|>help(solve)|
  \item \verb|>?solve|
\end{itemize}
对于一些特殊的字符串可以加上双引号,如: \verb|>help("[[")|


%\begin{verbatim}
%git config --global http.postBuffer  524288000
%\end{verbatim}

\end{document}
