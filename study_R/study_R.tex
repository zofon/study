\documentclass[a4paper,12pt]{ctexbook}
\usepackage[margin=2cm]{geometry}
\usepackage{graphicx}
\usepackage{subfigure}
\usepackage{amsmath}
\usepackage{float}
\usepackage[colorlinks,linkcolor=black]{hyperref}%colorlinks启用链接颜色,linkcolor指定对应的颜色
\pagestyle{empty}
\begin{document}

\begin{center}
\huge \textbf{R学习}
\end{center}

\tableofcontents
\newpage
\begin{flushleft}

\chapter{准备}
\section{R语言介绍}
R语言和其他的语言之间提供了非常好的接口。
\begin{itemize}
  \item R语言对大小写敏感。
  \item 基本的命令是表达式或者赋值。
  \item 命令可以被;隔开。
  \item 注释符号用\#
\end{itemize}

%暂时没有理解这句话。
R的缺点:
\begin{itemize}
  \item 耗内存,所以要用rm命令来删除对象,以释放内存,如:\verb|rm(x,y,z)|。
  \item 精度有问题
\end{itemize}

\section{R命令介绍}
在linux下面使用R的时候,我们一般用到的命令是:
\begin{verbatim}
R --vanilla <plot.R > a.out
\end{verbatim}
这里的 --vanilla 是参数,当然还有别的很多参数,具体有兴趣可以用man命令查看。
<plot.R 的意思是将plot-R这个文件作为一个输入。既然有输入,那自然有输出,>a.out 的意思就是将显示的内容输出到a.out这个文件中。

\section{关于包的安装与使用}
\subsection{安装对应的包}
R语言中,有多种方式可以导入数据包,如下,一目了然:
\begin{verbatim}
library(Hmisc)
source("plotter.R")
\end{verbatim}
如果没有对应的包,就要安装,其中source("plotter.R"),这个plotter.R文件是在当前的目录下的。
在linux下,进入R之后,用如下命令:
\begin{verbatim}
install.packages("ggplot2")
install.packages("Hmisc")
\end{verbatim}
\subsection{包的使用}
用source命令导入包,如:\verb|>source("plot.R")|

\section{查看帮助}
如查看solve的帮助:
\begin{itemize}
  \item \verb|>help(solve)|
  \item \verb|>?solve|
\end{itemize}
对于一些特殊的字符串可以加上双引号,如: \verb|>help("[[")|

\chapter{基础语法}
mode(X)\#可以查看变量的类型


\section{算术操作和向量运算}
创建含有5个值的向量x:\\
\begin{itemize}
	\item \verb|x<-c(10,2.5,3.4,2,6,1)|
	\item \verb|x=c(10,2.5,3.4,2,6,1)|
	\item \verb|assign("x",c(10,2.5,3.4,2,6,1))|
	\item \verb|c(10,2.5,3.4,2,6,1)->x|
\end{itemize}
下面是一些简单的应用:
\begin{itemize}
	\item \verb|1\x|\#显示x的倒数。
	\item \verb|y<-c(x,0,x)|\#创建y向量。
	\item \verb|v<-2*x+y+1|\#
\end{itemize}

接下来是一些常用的数学函数:
\begin{description}
	\item[log(X)] log函数
	\item[exp(X)] 以e为底的指数函数
	\item[sin(X)] sin
	\item[cos(X)] cos
	\item[tan(X)] tan
	\item[sqrt(X)] 对里面的数开根号
	\item[max(X)]
	\item[min(X)]
	\item[length(X)]
	\item[sum(X)]
	\item[prod(X)] 得到向量中所有数的乘积
	\item[mean(X)] 得到均值
	\item[var(X)] 得到方差
	\item[sort(X)] 对X进行排序
	\item[rev(X)] 颠倒向量
	\item[sd(X)] 得到标准差
%	\item[] describe
\end{description}
\section{向量的组合}
X1=c(1,2,3,4)\\
X2=c(5,6,7,8)

rbind(X1,X2)\#得到一个排列的矩阵
$$\begin{bmatrix}
	1&2&3&4 \\
	5&6&7&8 
\end{bmatrix}$$
cbind(X1,X2)\#另一种排列方式	
$$\begin{bmatrix}
1&5\\
2&6\\
3&7\\
4&8
\end{bmatrix}$$

\section{字符和字符向量}
字符向量既可以用双引号也可以使用单引号。

\verb|\n|\#换行
\verb|\t|\#制表符
\verb|\b|\#退格

c()可以将几个字符向量链接成一个字符向量。\\
paste()可以进行任意的链接。

letters,这是一个特殊的向量,里面包含26个字母。例如letters[2]='b'

\section{关于正则序列}
1:30等价于c(1,2,...,29,30)\#请注意,冒号:的优先级别是最高的。\\
30:1也是同样的道理,可以产生逆向序列。

seq(2,100,by=2)\#指定公差,表示(2,4,...,98,100),by就公差的意思。\\
seq(5,121,length=10)\#指定长度。

a[i]表示a向量中的第i个元素。\\
a[2,3,4]无法显示,报错如下:\verb|Error in a[2, 4] : incorrect number of dimensions|\\
a[2:4]表示a向量中第二和到第四个元素,返回的是一个3个数值的向量。

a[-1]表示第一个不显示。\\
a[-(1:3)]表示第一个到第三个不显示。

\section{一些奇怪的函数}
函数is.na(X1)表示返回一个和X1长度相同的向量,里面的值为FALSE。

which.max(a)\#显示a向量中最大值的下标,不可以用对字符向量进行该操作。\\
which.min(a)

\section{矩阵的操作}
a1=c(1:12)\\
matrix(a1,nrow=3,ncol=4),\\
matrix(a1,nrow=3,ncol=4,byrow=T)显示如下:
\begin{verbatim}
> a1<-c(1:12)
> a1
[1]  1  2  3  4  5  6  7  8  9 10 11 12
> matrix(a1,nrow=3,ncol=4)
[,1] [,2] [,3] [,4]
[1,]    1    4    7   10
[2,]    2    5    8   11
[3,]    3    6    9   12
> matrix(a1,nrow=3,ncol=4,byrow=T)
[,1] [,2] [,3] [,4]
[1,]    1    2    3    4
[2,]    5    6    7    8
[3,]    9   10   11   12
\end{verbatim}

\subsection{矩阵乘法}
\verb|a%*%b|注意,中间没有空格。
\begin{verbatim}
> a1%*%a1
[,1]
[1,]  650
> a1%*%t(a1)
[,1] [,2] [,3] [,4] [,5] [,6] [,7] [,8] [,9] [,10] [,11] [,12]
[1,]    1    2    3    4    5    6    7    8    9    10    11    12
[2,]    2    4    6    8   10   12   14   16   18    20    22    24
[3,]    3    6    9   12   15   18   21   24   27    30    33    36
[4,]    4    8   12   16   20   24   28   32   36    40    44    48
[5,]    5   10   15   20   25   30   35   40   45    50    55    60
[6,]    6   12   18   24   30   36   42   48   54    60    66    72
[7,]    7   14   21   28   35   42   49   56   63    70    77    84
[8,]    8   16   24   32   40   48   56   64   72    80    88    96
[9,]    9   18   27   36   45   54   63   72   81    90    99   108
[10,]   10   20   30   40   50   60   70   80   90   100   110   120
[11,]   11   22   33   44   55   66   77   88   99   110   121   132
[12,]   12   24   36   48   60   72   84   96  108   120   132   144
\end{verbatim}

%\begin{verbatim}
%git config --global http.postBuffer  524288000
%\end{verbatim}
\end{flushleft}
\end{document}
