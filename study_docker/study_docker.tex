\documentclass[a4paper,12pt]{ctexart}
\usepackage[margin=2cm]{geometry}
\usepackage{graphicx}
\usepackage{subfigure}
\usepackage{float}
\usepackage{url}
\usepackage[colorlinks,linkcolor=black]{hyperref}%colorlinks启用链接颜色,linkcolor指定对应的颜色
\pagestyle{empty}
\CTEXsetup[format={\Large\bfseries}]{section}%可以让section的标题左对齐。
%\CTEXsetup[format+={\flushleft}]{section}%让section的标题居左
%\renewcommand{\thesection}{\chinese{section}}%将“1.1”改为汉字“一”,但是subsection就会变成  六.1 ,比较难看,还是不用比较好。
\begin{document}

\begin{center}
学习Docker
\end{center}

%\tableofcontents
%\newpage

\section{安装Docker~\cite{安装Docker步骤}}
\begin{itemize}
	\item \#useradd docker -g docker %创建docker用户并将其纳入docker这个用户组 
	\item \#sudo apt-get update %更新源
	\item \#sudo apt-getinstall linux-image-generic-lts-trusty %安装依赖包
	\item \#sudo reboot
	\item \#wget -qO- https://get.docker.com/ | sh %获取最新版本的Docker
	\item \#sudo docker run hello-world %验证安装是否成功
\end{itemize}

\section{Docker相关命令}
\section{例子}

\bibliographystyle{plain}
\bibliography{Referenzarchiv}

%\bibliographystyle{IEEEtran.bst}
%\bibliographystyle{plain}
%表示指定文献引用的格式设置参考文献的类型 (bibliography style). 标准的为 plain:
%其它的类型包括:
%unsrt – 基本上跟 plain 类型一样, 除了参考文献的条目的编号是按照引用的顺序, 而不是按照作者的字母顺序.
%alpha – 类似于 plain 类型, 当参考文献的条目的编号基于作者名字和出版年份的顺序.
%abbrv – 缩写格式 .

%\bibliography{ReferenzarchivWithoutURLs,OtherReferences}
%对应的引用文件。
\end{document}
